\documentclass[]{corona-b5-1.1}%tombow

\usepackage[dvipdfmx]{graphicx}
\usepackage{makeidx,bigstrut,multirow,fancybox,color,amsmath,framed,amsthm}

%%%begin yasuda definition%%%
\usepackage{verbatim, setspace, ascmac}
\usepackage{listings, jlisting}%jlistingsはtexliveにはないので,ダウンロード必要。
%% https://osdn.net/projects/mytexpert/downloads/26068/jlisting.sty.bz2/
\lstset{%
	language={Python},
	basicstyle={\small},%
	identifierstyle={\small},%
	commentstyle={\small\itshape},%
	keywordstyle={\small\bfseries},%
	ndkeywordstyle={\small},%
	stringstyle={\small\ttfamily},
	frame={tb},
	breaklines=true,
	columns=[l]{fullflexible},%
	numbers=left,%
	xrightmargin=0zw,%
	xleftmargin=3zw,%
	numberstyle={\scriptsize},%
	stepnumber=1,
	numbersep=1zw,%
	lineskip=-0.5ex%
}

%%%end yasuda definishon%%%


\begin{document}
\setcounter{chapter}{0}
\setcounter{section}{0}

\section{latex テスト}

\subsection{listings.sty, jlistingを用いたソースコードの参照}
次のコードのように直接記述することができる
\begin{lstlisting}[caption=test, label=test]
Print("Hello World!")
def add(a, b):
    return a + b
c = add(1, 2)
\end{lstlisting}

もしくは,\verb|\lstinputlisting{filepath}|を用いて外部ファイルを参照することができる。
\lstinputlisting[caption=sim01\_simple.py, label=sim01]{../../ch4_3/sim01_simple.py}

\end{document}
